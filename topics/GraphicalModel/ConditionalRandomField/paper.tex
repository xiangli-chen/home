%sample file: sampart.tex
% The sample article for the amsart document class

\documentclass{amsart}
\usepackage{amssymb,latexsym,graphicx}

\theoremstyle{plain}
\newtheorem{theorem}{Theorem}
\newtheorem{corollary}{Corollary}
\newtheorem*{main}{Main~Theorem}
\newtheorem{lemma}{Lemma}
\newtheorem{proposition}{Proposition}

\theoremstyle{definition}
\newtheorem{definition}{Definition}

\theoremstyle{remark}
\newtheorem*{notation}{Notation}

\numberwithin{equation}{section}

\newcommand{\bx}{{\bf x}}
\newcommand{\by}{{\bf y}}
\newcommand{\br}{\boldsymbol{\lambda}}


\begin{document}
\title[sub-title]
      {Notes of Conditional Random Field}
\author{Xiangli Chen}
\address{Computer Science Department\\
         University of Illinois at Chicago\\
         Chicago, IL 60607} 
\email{xchen40@uic.edu}
\urladdr{https://www.cs.uic.edu/~xchen/}
%\thanks{thanks.}  
%\keywords{keywords} 
%\subjclass[2000]{Primary: 06B10; Secondary: 06D05}
\date{\today}
%\begin{abstract}
%   abstract \emph{emphasize}
%\end{abstract}
\maketitle
%\begin{scriptsize}
%{\footnotesize •}
%\end{scriptsize}
\section{Introduction}\label{sec:intro}
\noindent HMMs and stochastic grammars \cite{lafferty2001conditional} are generative models, 
assigning a joint probability to paired observation and label sequences. 
A generative model needs to enumerate all possbile observation sequences 
that the inference problem for such model is intractable. 
This difficulty is one of the main motivation for looking at conditinal models as an alternative.
A conditional model does not expand modeling effort on the ovservations 
and the conditional probability of the label sequence can depend on arbitrary, 
non-independent features of the observation sequence without forcing the model 
to account for the distribution of those dependencies. 
Maximum entropy Markov models (MEMMs) \cite{mccallum2000maximum} are conditional probabilistic sequence models. 
But like other non-generative finite-state models based on next-state classifiers, 
such as discriminative markov models, MEMMs has the label bias problem: 
the transitions leaving a given state compete only against each other, 
rather than against all other transitions in the model.
\section{Conditional Random Field}\label{sec:intro}
\noindent {\bf Conditonal random field} (CRFs)\cite{lafferty2001conditional} is a sequence modeling framework 
that has all the advantages of MEMMs but also solves the label bias problem in a principled way. 
CRFs perform better than HMMs and MEMMs when the true data distribution has higher-order dependencies than the model, 
as is ofter the case in practice.
If the graph $G = (V,E)$ of $Y$ is a tree and the joint distribution is positive, 
by Hammersley Clifford Theorem\cite{hammersley1971markov}, the conditional model takes the form: 
\[p(Y|X) = \frac{p(X,Y)}{p(X)} = \frac{1}{P(X)Z}\phi(X)\prod_{e\in E}\phi_e(Y_e,X)\prod_{v\in V}\phi_v(Y_v,X).\]
where $Z$ is the normalized constant with respect to the joint distribution.
We can also denote it to be:
\[p({\bf y}|{\bf x}) = \frac{1}{Z({\bf x})}\exp\left(\sum_{e\in E}\psi_{e}({\bf y}_e,
{\bf x})+\sum_{v\in V}\psi_{v}({\bf y}_v,{\bf x})\right)\]
In addition, we assume the conditional model is a probability distribution of exponential families,
\[p({\bf y}|{\bf x}) = \exp(\phi({\bf x},{\bf y})\cdot \theta-g(\theta|{\bf x})).\]
Apply Hammersley Clifford Theorem for exponential families,
\[p({\bf y}|{\bf x}) = \frac{1}{Z({\bf x})}\exp\left(\sum_{e\in E}\phi_e({\bf y}_e,
{\bf x})\cdot \theta_{e}+\sum_{v\in V}\phi_v({\bf y}_v,{\bf x})\cdot\theta_{v}\right).\]
We consider a chain-structured CRFs with label sequence $Y = (Y_1,Y_2,...,Y_n)$. 
The cliques are $(x,y_i)$ and $y_{i-1},y_i$'s. 
In general, CRFs assume the unknown parameter does not depend on the position of sequence. 
As a consequence, the conditional model takes the following form:
\[p({\bf y}|{\bf x})=\frac{1}{Z({\bf x})}\exp\left(\sum_{i=1}^n
\left(\sum_j\lambda_j t_j(y_{i-1},y_i,{\bf x},i)+\sum_k \mu_k s_k(y_i,{\bf x},i)\right)\right)\]
where $t_j$'s and $s_k$'s are feature functions.
Note that the cliques are $(x,y_i)$ and $y_{i-1},y_i$'s or more precisely formulation would be:
\[p({\bf y}|{\bf x})=\frac{1}{Z({\bf x})}\exp\left(\sum_{i=1}^n
\left(\sum_j\lambda_j t_j(y_{i-1},y_i)+\sum_k \mu_k s_k(y_i,x_i)\right)\right).\]
To obtain a more uniform notation, we specifies a feature vector ${\bf f}({\bf y},{\bf x},i)$. 
An element of the feature vector $f_j({\bf y},{\bf x},i)$ is either a state feature $s_j(y_i,{\bf x},i)$ 
or transitive feature $t_j(y_{i-1},y_i,{\bf x},i)$. 
(In general, we let $t_j(y_{i-1},y_i,{\bf x},i)=0$, when $i=1$.)
Let,\[F_j({\bf y},{\bf x})=\sum_{i=1}^n f_j({\bf y},{\bf x},i).\]
Then a general conditional model for CRFs is,
\[p({\bf y}|{\bf x})=\frac{1}{Z({\bf x})}\exp\left(\sum_j\lambda_jF_j({\bf y},
{\bf x})\right)=\frac{1}{Z({\bf x})}\exp\left(\boldsymbol{\lambda} \cdot{\bf F}({\bf y},{\bf x})\right).\]
We can train a CRF\cite{sha2003shallow} by maximizing the log-likelihood of a given training set 
$T=\{({\bf x}_k,{\bf y}_k)\}_{k=1}^N$.
\[\mathcal{L}_{\boldsymbol{\lambda}}=\sum_k \log p_{\boldsymbol{\lambda}}({\bf y}_k|{\bf x}_k)\]
The gradient with respect to $\boldsymbol{\lambda}$,
\[\nabla \mathcal{L}_{\boldsymbol{\lambda}}=\sum_k \left[{\bf F}({\bf y}_k,{\bf x}_k)
-\mathbb{E}_{p_{\boldsymbol{\lambda}}({\bf y}|{\bf x}_k)}{\bf F}({\bf Y},{\bf x}_k)\right].\]
\section{Dynamic Programming}
\noindent The chanllenging part of computing the gradient of CRF is 
$\mathbb{E}_{p_{\boldsymbol{\lambda}}({\bf y}|{\bf x}_k)}{\bf F}({\bf Y},{\bf x}_k)$. 
Here we use dynamic programming method.
\noindent Let $y\in \mathcal{Y}$, given $\bf x$, define $n$ transition matrix
$\{{\bf M}_i(\mathcal{Y},{\bf x})|i=1,...,n\}$, where each
${\bf M}_i(\mathcal{Y},\bf x)$ is a $|\mathcal{Y}\times\mathcal{Y}|$
matrix with elements of the form:
\[{\bf M}_i[y_{i-1}=y,y_i=y']=\exp\left({\boldsymbol\lambda}^T{\bf f}(y_{i-1}=y,y_i=y',{\bf x},i)\right)\]
\noindent For each $\lambda_j$, define $n$ feature matrix $\{{\bf Q}_{ji}(\mathcal{Y},{\bf x})|i=1,..,n\}$, 
where each ${\bf Q}_{ji}(\mathcal{Y},{\bf x})$ is a $|\mathcal{Y}\times\mathcal{Y}|$ 
matrix with elements of the form:
\[{\bf Q}_{ji}[y_{i-1}=y,y_i=y']=f_j(y_{i-1}=y,y_i=y',{\bf x},i)\]
Note that $F_j(\by,\bx)=\sum_{i=1}^n f(y_{i-1},y_i,\bx,i)$. Then,
\begin{align*}
\mathbb{E}_{p_{\br}(\by|\bx_k)}F_j({\bf Y},\bx_k)&=\sum_{\by\in\mathcal{Y}}p_{\br}(\by|\bx_k)F_j(\by,\bx_k)\\
&=\sum_{i=1}^n\frac{{\bf \alpha}_{i-1}^T({\bf Q}_{ji}\circ{\bf M}_i){\bf \beta}_i}{Z_{\br}(\bx)}
\end{align*}
where $\circ$ is the element-wise matrix product, $Z_{\br}(\bx)$
is the norm constant, $\alpha_i$ and $\beta_i$ is
defined by:
\begin{align}
\alpha_i^T = \begin{cases}
\alpha_{i-1}^T{\bf M}_i & 0< i \le n\\
{\bf 1}^T & i=0
\end{cases}. \notag
\end{align} 
\begin{align}
\beta_i = \begin{cases}
{\bf M}_{i+1}\beta_{i+1} & 0\le i < n\\
{\bf 1} & i=n
\end{cases}. \notag
\end{align}
Note that $Z_{\br}(\bx)=\alpha_n^T{\bf 1}$. 
\bibliography{biblio}{}
%\bibliographystyle{apalike}
\bibliographystyle{plain}	
\end{document}

