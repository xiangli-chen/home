\documentclass{article}

% if you need to pass options to natbib, use, e.g.:
% \PassOptionsToPackage{numbers, compress}{natbib}
% before loading nips_2016
%
% to avoid loading the natbib package, add option nonatbib:
% \usepackage[nonatbib]{nips_2016}

\usepackage[final]{nips_2016_note}

% to compile a camera-ready version, add the [final] option, e.g.:
% \usepackage[final]{nips_2016}

\usepackage[utf8]{inputenc} % allow utf-8 input
\usepackage[T1]{fontenc}    % use 8-bit T1 fonts
\usepackage{hyperref}       % hyperlinks
\usepackage{url}            % simple URL typesetting
\usepackage{booktabs}       % professional-quality tables
\usepackage{amsfonts}       % blackboard math symbols
\usepackage{nicefrac}       % compact symbols for 1/2, etc.
\usepackage{microtype}      % microtypography

\usepackage{cite}

\newtheorem{corollary}{Corollary}
\newtheorem{definition}{Definition}
\newtheorem{theorem}{Theorem}
\newtheorem{lemma}{Lemma}

\newcommand{\argmin}{\operatornamewithlimits{argmin}}
\newcommand{\argmax}{\operatornamewithlimits{argmax}}

%\title{Formatting template for reading notes using NIPS 2016 style}
\title{Notes of Conditional Random Field}

% The \author macro works with any number of authors. There are two
% commands used to separate the names and addresses of multiple
% authors: \And and \AND.
%
% Using \And between authors leaves it to LaTeX to determine where to
% break the lines. Using \AND forces a line break at that point. So,
% if LaTeX puts 3 of 4 authors names on the first line, and the last
% on the second line, try using \AND instead of \And before the third
% author name.

\author{
  Xiangli Chen
  \thanks{https://www.cs.uic.edu/~xchen/}\\
  Department of Computer Science\\
  University of Illinois at Chicago\\
  Chicago, IL 60607 \\
  \texttt{xchen@uic.edu} \\
}

\begin{document}
% \nipsfinalcopy is no longer used

\maketitle

\date{\today}

\begin{abstract}
abstract

\end{abstract}

\section{Introduction}\label{sec:intro}
\noindent HMMs and stochastic grammars \cite{lafferty2001conditional} are generative models, 
assigning a joint probability to paired observation and label sequences. 
A generative model needs to enumerate all possbile observation sequences 
that the inference problem for such model is intractable. 
This difficulty is one of the main motivation for looking at conditinal models as an alternative.
A conditional model does not expand modeling effort on the ovservations 
and the conditional probability of the label sequence can depend on arbitrary, 
non-independent features of the observation sequence without forcing the model 
to account for the distribution of those dependencies. 
Maximum entropy Markov models (MEMMs) are conditional probabilistic sequence models. 
But like other non-generative finite-state models based on next-state classifiers, 
such as discriminative markov models, MEMMs has the label bias problem: 
the transitions leaving a given state compete only against each other, 
rather than against all other transitions in the model.
{\small
\bibliography{biblio}
\bibliographystyle{plain} 
}
\end{document}
